\section{The archive} 

At the time of writing, the full ULTRACAM archive is still undergoing analysis by the automated pipeline. At present \emph{159} nights, out of a total of \emph{390}, have been processed and are available for viewing at the following URL: \url{http://url.to.be.provided/}. Of these processed nights, approximately \emph{xx} have been investigated for variability. At the moment, the investigation is almost purely visual, based on a human inspecting each light curve. The web interface is designed so that it is easy for the viewer to examine the light-curves of all of the objects systematically. The user-interface allows the users to see each light curve, one-by-one by press the 'right' and ;left'  arrow keys on the keyboard. More information on how to use this interface can be found in the User Manual~\ref{chap:usermanual}. 

\section{Data quality}

Analysis of the data quality (eg accuracy of photometry, WCS solutions, tracking of objects, distortion of the fields). Include a discussion on the quality of the photometry compared to the current Ultracam pipeline. How does 'tweaking' of the SExtractor parameters effect the quality of the data? 

\section{Highlights}

Some 'interesting' objects discovered by the pipeline.


