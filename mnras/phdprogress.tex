% mnras_template.tex
%
% LaTeX template for creating an MNRAS paper
%
% v3.0 released 14 May 2015
% (version numbers match those of mnras.cls)
%
% Copyright (C) Royal Astronomical Society 2015
% Authors:
% Keith T. Smith (Royal Astronomical Society)

% Change log
%
% v3.0 May 2015
%    Renamed to match the new package name
%    Version number matches mnras.cls
%    A few minor tweaks to wording
% v1.0 September 2013
%    Beta testing only - never publicly released
%    First version: a simple (ish) template for creating an MNRAS paper

%%%%%%%%%%%%%%%%%%%%%%%%%%%%%%%%%%%%%%%%%%%%%%%%%%
% Basic setup. Most papers should leave these options alone.
\documentclass[a4paper,fleqn,usenatbib]{mnras}

% MNRAS is set in Times font. If you don't have this installed (most LaTeX
% installations will be fine) or prefer the old Computer Modern fonts, comment
% out the following line
\usepackage{newtxtext,newtxmath}
% Depending on your LaTeX fonts installation, you might get better results with one of these:
%\usepackage{mathptmx}
%\usepackage{txfonts}

% Use vector fonts, so it zooms properly in on-screen viewing software
% Don't change these lines unless you know what you are doing
\usepackage[T1]{fontenc}
\usepackage{ae,aecompl}


%%%%% AUTHORS - PLACE YOUR OWN PACKAGES HERE %%%%%
% \usepackage(graphicx,tabularx}

% Only include extra packages if you really need them. Common packages are:
\usepackage{graphicx}	% Including figure files
\usepackage{amsmath}	% Advanced maths commands
\usepackage{amssymb}	% Extra maths symbols

%%%%%%%%%%%%%%%%%%%%%%%%%%%%%%%%%%%%%%%%%%%%%%%%%%

%%%%% AUTHORS - PLACE YOUR OWN COMMANDS HERE %%%%%

% Please keep new commands to a minimum, and use \newcommand not \def to avoid
% overwriting existing commands. Example:
%\newcommand{\pcm}{\,cm$^{-2}$}	% per cm-squared

%%%%%%%%%%%%%%%%%%%%%%%%%%%%%%%%%%%%%%%%%%%%%%%%%%

%%%%%%%%%%%%%%%%%%% TITLE PAGE %%%%%%%%%%%%%%%%%%%

% Title of the paper, and the short title which is used in the headers.
% Keep the title short and informative.
\title[Richard Ashley - PhD project]{PhD Progress report - Richard Ashley}

% The list of authors, and the short list which is used in the headers.
% If you need two or more lines of authors, add an extra line using \newauthor
\author[R.P. Ashley et al.]{
R.P. Ashley$^{1}$\thanks{E-mail: r.p.ashley@warwick.ac.uk}\\
% List of institutions
$^{1}$Department of Physics, University of Warwick, Gibbet Hill Road, Coventry, CV4 7AL, UK\\
}

% These dates will be filled out by the publisher
\date{4 August 2015}

% Enter the current year, for the copyright statements etc.
\pubyear{2015}

% Don't change these lines
\begin{document}
\label{firstpage}
\pagerange{\pageref{firstpage}--\pageref{lastpage}}
\maketitle


%%%%%%%%%%%%%%%%%%%%%%%%%%%%%%%%%%%%%%%%%%%%%%%%%%

%%%%%%%%%%%%%%%%% BODY OF PAPER %%%%%%%%%%%%%%%%%%

\section{Outline of the project}
Recent large scale synoptic surveys such as the Catalina Real-time Transit Survey (CRTS)  and the Pan-STARRS survey have yielded a large number of newly discovered transient objects. Many of these objects are compact binaries that exhibit transient phenomena due to the nature of the interaction of the components of the binary. This list includes cataclysmic variable stars that undergo outbursts, binaries that eclipse and pairs of stars that show modulation in their light-curves due to their orbit around each other. Understanding the evolution of these objects benefits from having a large population sample such that the broader statistics of the overall population can be modelled. This is used to place constraints on masses of the components, orbital mechanics and the relative time spent in each phase of the evolution of these systems. Objects detected in these large-scale surveys need to be studied in more detail through having their light-curves resolved with a higher time resolution and with more complete coverage than the surveys can provide. These photometric light-curves will reveal eclipses, phased modulations and timing variations and therefore allow the resolution of the orbital parameters of the systems. 

The University of Warwick has recently acquired a 1 metre telescope located at the Roque de los Muchachos Observatory. This telescope will be used to follow up new transient objects detected in current surveys such as, but not limited to, Pan-STARRS and CRTS. Since the telescope is not fully commissioned at the moment, a large portion of the PhD project will be the undertaking of the commissioning of the telescope including creation of the software required to operate the telescope remotely and robotically. 

\section{Review of the research field}

\subsection{Compact objects found in surveys}
\citet{Breedt2014} demonstrate the utility of large scale surveys in producing a list of transient objects worthy of follow up study. In this case the Catalina Real Time Survey (CRTS), \citep{Drake2009}. The original and primary purpose of CRTS is to detect near-Earth objects such as asteriods that are of interest to solar system astronomers and could be important for an early warning of any objects on a potential collision course with Earth. More generally, the survey has been used to identify many kinds of optical transients that vary on timescales from minutes to years. The most recent catalogue of identified transients can be found in \citet{CatalinaCatalog}. So far, CRTS has detected more than 9000 transient objects since its start of operations in 2007. The survey makes use of a network of  small telescope situation in yy locations around the world. The survey covers an area of the sky of $\sim 30,000\,deg^2$ between $-75^o < \delta < 65^o$ excluding an area of a few degrees near the galactic plan and revisits most locations many times per year with a typical cadence of around 2 weeks. The limiting magnitude of the survey is about $19-21$, depending on the telescope. Details of the telescopes and the observing program are described in \citet{Drake2009}. 

\subsubsection{Cataclysmic Variables}
Cataclysmic Variables (CVs) is just one active area of research that can benefit from catalogues such as this one provided by CRTS. \citet{Breedt2014} use the catalogue to detect and investigate the population of CVs pulled from CRTS. Understanding a broader population of these objects enables the study of binary evolution of systems like this. CVs are compact binary objects that consist of a White Dwarf that is in close orbit with a main sequence companion and is accreting matter from that companion. During the lifetime of the CV the orbital period is expected to evolve. Since the orbital period is the most easily observable property of the system, it provides an important indicator to our understanding of the evolution of the system. We cannot usually observe the period changing on human timescales of tens of years so we need a larger population sample that can give us overview of the various stages of the evolution of the orbit. The orbits of CVs evolve from a long period (several hours) down to a period minimum of about 80 minutes. At this point the donor star will have lost enough mass such that it no longer contains enough to continue hydrogen burning at its core. The thermal timescale of the donor star, which describes how quickly its radius will contract in response to the mass loss, becomes longer than the mass loss itself and the star expands beyond the Roche lobe. Now the dynamics of the system force the separation of the two stars to increase and therefore period to increase. As the system approaches this period minimum, the accretion rate slows down, events take longer and therefore, in a large population sample, we would expect a build-up of systems with periods at around 80 minutes. So far, it has been difficult to confirm this period spike for CVs.  CVs selected through photometric variability will suffer from a selection bias since those with more frequent outbursts are more likely to be discovered. CVs close to the period minimum are slower accretors and therefore less likely to go into outburst. The period spike has only recently been confirmed from analysis by \citet{Gaensicke2009} of the CVs found in the Sloan Digital Sky Survey (SDSS).  Since, G{\"a}nsicke selected the CV sample through inspection of the spectra, he was able to avoid the selection bias. 

An example of a compact binary for which a quick follow-up on the one metre would be useful is a CV that goes into outburst. Also known as a dwarf novae, a CV in outburst is one in which the accretion disc is undergoing a temporary thermal instability. As the disc becomes more viscious thermal energy is radiated across its surface. The overall brightness of the system increases by 2 - 6 magnitudes, which corresponds to an increase of from 6 to 250 times its output in quiescence.  The onset of outburst occurs on the timescale of around one day and the outburst itself can last several days or weeks. In some cases the disc forms an assymetrical component during the outburst and this precesses around the system with a period that is slightly different to the orbital period. Since these phenomena are transient in nature it is useful to be able to follow up on the detection of an outburst within a day or so. 

\subsubsection{Polars}
CVs that include white dwarf stars with strong magnetic fields define a sub-class known as Polars. Polars are so-called because the optical radiation coming from the object is strongly polarised. Matter accreting from the donor star is expected to form an accretion disc around the White Dwarf. This disc is one of the dominant features in the study of CVs. Since the accretion stream is composed of hot hydrogen gas, in free fall that is heating up to beyond ionisation temperature it becomes a plasma that can be affected by magnetic fields. Therefore, if the White Dwarf's magnetic field is strong enough, it can divert the flow of the stream such that it now follows the field lines and, rather than forming a disc in the plane of the CV's orbit, the stream diverts to accrete directly on to one or both of the CVs magnetic poles. 

\subsection{Optical transient surveys}

\subsubsection{GAIA}
The GAIA satellite was launched in 2013. Over the course of the next five or so years, it will measure the position and magnitude of about 1 billion stars about 70 times each. This will provide a large source of optical transients. Detected transients are published via a system called the GAIA alerts system  \footnote{http://www.gaia.ac.uk/selected-gaia-science-alerts}. This system will provide a live source of objects that might be compact binaries and worthy of follow up in this project. An aim of this PhD project is to implement a GAIA allerts follow up service with the newly commissioned Warwick 1-metre telescope.  At the moment, these alerts are not live, but are expected to come online in November 2015. The initial time delay from object detection to release of the alert is undefined as the identification pipeline is still in development. However, we expect that, once the pipeline is established, alerts will be released in near real-time (minutes to hours) after detection. The Warwick 1 metre telescope will then add any relevant alerts to its own observing schedule.  


\subsubsection{LSST}
The Large Synoptic Survey Telescope (LSST) is a new all-sky survey that is expected to go live 2023. 

\subsubsection{ASAS - SN}

\section{Research Progress}

\section{Telescope Automation Progress}

\section{Key text review}

\section{Conclusions}

Conclusions....

%%%%%%%%%%%%%%%%%%%%%%%%%%%%%%%%%%%%%%%%%%%%%%%%%%

%%%%%%%%%%%%%%%%%%%% REFERENCES %%%%%%%%%%%%%%%%%%

% The best way to enter references is to use BibTeX:

\bibliographystyle{mnras}
\bibliography{../rashley} % if your bibtex file is called example.bib


% Alternatively you could enter them by hand, like this:
% This method is tedious and prone to error if you have lots of references
%\begin{thebibliography}{99}
%\bibitem[\protect\citeauthoryear{Author}{2012}]{Author2012}
%Author A.~N., 2013, Journal of Improbable Astronomy, 1, 1
%\bibitem[\protect\citeauthoryear{Others}{2013}]{Others2013}
%Others S., 2012, Journal of Interesting Stuff, 17, 198
%\end{thebibliography}

%%%%%%%%%%%%%%%%%%%%%%%%%%%%%%%%%%%%%%%%%%%%%%%%%%

%%%%%%%%%%%%%%%%% APPENDICES %%%%%%%%%%%%%%%%%%%%%

%\appendix

%\section{Some extra material}

%If you want to present additional material which would interrupt the flow of the main paper,
%it can be placed in an Appendix which appears after the list of references.

%%%%%%%%%%%%%%%%%%%%%%%%%%%%%%%%%%%%%%%%%%%%%%%%%%


% Don't change these lines
\bsp	% typesetting comment
\label{lastpage}
\end{document}

% End of mnras_template.tex
