
\section{Current status of the pipeline}

Where it works well and how it is useful to the researchers using ULTRACAM.
\begin{itemize}
  \item Fully automated. One command issued at the end of the night's observing will produce a browseable page. (ie \emph{daybuilder.py 2014-08-15})
  \item Ability to browse the (nearly) complete ULTRACAM data from anywhere with an Internet connection. 
  \item Light-curves for 10,000s of objects.
\end{itemize}

Where it fails or needs improvement.
\begin{itemize}
  \item Astrometry
  \item High cadence runs
  \item Out of focus runs
  \item Memory problems in the browser for large datasets 
\end{itemize}

\section{Recommendations for ULTRACAM users}

Things to remember for ULTRACAM users. 
\begin{itemize}
	\item Accurate entry of the target info.
	\item Don't rotate the camera unless you have to.
	\item Large fields of view if you can. Don't worry about data size. 
\end{itemize} 

\section{Next steps}
\begin{itemize}
	\item Server-side and client-side components for the browser interface to remedy memory problems.
	\item Automatic variability detection and light-curve classification. 
	\item Investigate source-extraction alternatives to cope with crowded fields and out-of-focus runs. 
\end{itemize} 
 

