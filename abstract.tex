Since May 2002, the  Ultracam high-speed CCD photometric camera 
(ULTRACAM) has been taking observations of a variety of astronomical 
objects using the William Herschel Telescope (WHT), Very Large 
Telescope (VLT) and New Technology Telescope (NTT). Over this period 
it has generated approximately 10 Terabytes of raw data (CCD frames 
plus meta-data) taken on 406 observing nights and covering 
approximately 566 target objects. In this data there may be objects 
that may hold scientific interest but have not been investigated since they were not the intended target object for the observer. 
Objects in the data need to be identified, listed and have 
reductions performed to determine their light-curves. The goal of 
this 1 year MSc project is to build an automated software pipeline 
that will process the ULTRACAM data archive and produce a 
browse-able version of the light-curves and meta-data for all of the 
observations. 
