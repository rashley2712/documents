Since May 2002, the  Ultracam high-speed CCD photometric camera (ULTRACAM) has been taking observations of a variety of astronomical objects using the William Herschel Telescope (WHT), Very Large Telescope (VLT) and New Technology Telescope (NTT). Over this period it has produced approximately 10 Terabytes of raw data (CCD frames plus meta-data) taken on 406 observing nights and covering approximately 566 target objects. In these data there may be objects that hold scientific interest but have not been investigated since they were not the intended target object of the observer. Objects in the data need to be identified, listed and have reductions performed to determine their light-curves. In this project we have built a suite of software that is able to automatically reduce the full set of data residing in the ULTRACAM archive and produce light-curves for all of the objects identified in each run. We have compared the photometry to that produced by the current ULTRACAM pipeline and shown that our automated pipeline performs similarly. The reduced data have been made available via a set of interactive web pages allowing users to browse and review the archive. So far we have visually inspected the light-curves of about 20\% of the objects and have found several variables that are not listed in any catalog. These are {W UMa} stars, $\delta$ Scuti stars and eclipsing binaries. We have also found the rotation period of an asteroid found passing through the field during an ULTRACAM run. 
